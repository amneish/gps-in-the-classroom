\documentclass[conference]{IEEEtran}
\IEEEoverridecommandlockouts
% The preceding line is only needed to identify funding in the first footnote. If that is unneeded, please comment it out.
\usepackage{cite}
\usepackage{amsmath,amssymb,amsfonts}
\usepackage{algorithmic}
\usepackage{graphicx}
\usepackage{textcomp}
\usepackage{xcolor}
\def\BibTeX{{\rm B\kern-.05em{\sc i\kern-.025em b}\kern-.08em
    T\kern-.1667em\lower.7ex\hbox{E}\kern-.125emX}}
\begin{document}

\title{GPS in the Classroom}

\author{\IEEEauthorblockN{Andrew Neish}
\IEEEauthorblockA{\textit{Aeronautics and Astronautics} \\
\textit{Stanford University}\\
Stanford, USA \\
amneish@stanford.edu}
\and
\IEEEauthorblockN{Tyler Reid}
\IEEEauthorblockA{
\textit{Xona Space Systems}\\
San Mateo, USA \\
tyler@xonaspace.com}
\and
\IEEEauthorblockN{Frank van Diggelen}
\IEEEauthorblockA{\textit{Android Location} \\
\textit{Google}\\
Mountain View, USA \\
fvandiggelen@google.com}
\and
\IEEEauthorblockN{Grace Gao}
\IEEEauthorblockA{\textit{Aeronautics and Astronautics} \\
\textit{Stanford University}\\
Stanford, USA \\
gracegao@stanford.edu}
}

\maketitle

\section*{Abstract}

\begin{itemize}
    \item New tools available in the classroom with Android analysis tools
    \item Showcase some of the possibilities with these new products
    \item Structure of how Stanford teaches with the Android analysis tool
    \item Showcase student work that has been done 
    \item Student projects to present
    \begin{itemize}
        \item Brian Munguia's project on SVD
        \item NLOS project from Scott and Riley
        \item ...
    \end{itemize}
    \item Finish by talking about the merits of hands on learning
\end{itemize}

Teaching any subject can be a challenge, but teaching GPS comes with its own set of challenges.


\end{document}
