\documentclass[12pt, conference, onecolumn, draftclsnofoot]{IEEEtran}
\IEEEoverridecommandlockouts
% The preceding line is only needed to identify funding in the first footnote. If that is unneeded, please comment it out.
\usepackage{cite}
\usepackage{amsmath,amssymb,amsfonts}
\usepackage{algorithmic}
\usepackage{graphicx}
\usepackage{textcomp}
\usepackage{xcolor}
\def\BibTeX{{\rm B\kern-.05em{\sc i\kern-.025em b}\kern-.08em
    T\kern-.1667em\lower.7ex\hbox{E}\kern-.125emX}}
\begin{document}

\title{GNSS in the Classroom: Taking the Paralysis out of Analysis - Abstract}

\author{\IEEEauthorblockN{Andrew Neish}
\IEEEauthorblockA{\textit{Aeronautics and Astronautics} \\
\textit{Stanford University}\\
Stanford, USA \\
amneish@stanford.edu}
\and
\IEEEauthorblockN{Tyler Reid}
\IEEEauthorblockA{
\textit{Xona Space Systems}\\
San Mateo, USA \\
tyler@xonaspace.com}
\and
\IEEEauthorblockN{Frank van Diggelen}
\IEEEauthorblockA{\textit{Android Location} \\
\textit{Google}\\
Mountain View, USA \\
fvandiggelen@google.com}
\and
\IEEEauthorblockN{Grace Gao}
\IEEEauthorblockA{\textit{Aeronautics and Astronautics} \\
\textit{Stanford University}\\
Stanford, USA \\
gracegao@stanford.edu}
}

\maketitle

\iffalse
\begin{itemize}
    \item New tools available in the classroom with Android analysis tools
    \item Showcase some of the possibilities with these new products
    \item Structure of how Stanford teaches with the Android analysis tool
    \item Showcase student work that has been done 
    \item Student projects to present
    \begin{itemize}
        \item Brian Munguia's project on SVD
        \item NLOS project from Scott and Riley
        \item ...
    \end{itemize}
    \item Finish by talking about the merits of hands on learning
\end{itemize}
\fi

Teaching any subject can be difficult, but teaching GNSS offers a set of unique challenges.
GNSS as a technology represents one of the most complex and interdisciplinary undertakings ever achieved by mankind.
GNSS modernization today continues to push the boundaries of orbital mechanics, materials science, physics, rocket science, and a whole host of other cutting edge disciplines.
But for being such a complex system, GNSS has turned into one of the most recognized and used utilities throughout the globe.
Everyone has become an "Navigator" with their cell phone GNSS receiver at their fingertips.
So how does one boil all of this information down to one 10 week graduate course?
Furthermore, how does one go beyond giving students a background in GNSS and give them the tools they need to go out into the world and innovate in the world of navigation.

At Stanford, the GPS course has always thrived to achieve these two goals.
This is evident by the way in which this class has served as the inspiration to many to join the navigation profession and who today are leaders in the field of GNSS.
And now with the emergence of the GNSS Analysis Tools from Google (ref), there has never been more opportunity to give students a hands on and creative experience when learning about GNSS.
The GNSS Analysis Tools, now available on Android (ref url for GNSS Analysis Tools), allows anyone with an Android phone to record raw GNSS measurements and process them in endlessly creative ways.
By raw GNSS measurements, we're not simply referring to pseudoranges.
After all, a pseudorange is a derived measurement.
True raw measurements are the time stamps associated with satellite signal reception and transmission, AGC, $C$/$N_0$, and other measurements used to ultimately create a PVT solution.
With these measurements available to the masses, everyone can tinker and toy with real GNSS data and learn by going out into the real world and applying the knowledge the garnered in the classroom.

Over the years, the GPS course has been shaped in a way to engage students who come from different backgrounds and help them explore the intersection between their passions and GNSS.
The course is ten weeks long with each week covering a different part of the system. (show figure of the course here).
The class is now structured as a project based class where while the students are learning about the different subsystems that constitute GNSS, they are developing projects that push their understanding of GNSS to another level.
These projects can take inspiration from their current work and interests or they can be derived from any aspect of GNSS that piques their interest.
One of the most unique advantages afforded to those who are just beginning to learn the subject is a sort of innocent naievity.
This frees many students from imposing limits on their creativity of what they want to accomplish with a GNSS project and leads to innovative ideas that feel only accessible to those who haven't been immersed in the world of navigation for years. 
This paper showcases some of the projects that these students were able to accomplish in a mere matter of weeks.
Below are some example ideas and results carried out by the students.

\textit{\textbf{Showcase several projects here}}

The ultimate goal of this paper is to serve as an inspiration to those teaching GNSS.
Whether instructing undergraduates, graduates, or professionals, this paper outlines a hands on strategy useful to anyone wanting to learn more about GNSS.

\end{document}
