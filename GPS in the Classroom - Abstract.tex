\documentclass[conference]{IEEEtran}
\IEEEoverridecommandlockouts
% The preceding line is only needed to identify funding in the first footnote. If that is unneeded, please comment it out.
\usepackage{cite}
\usepackage{amsmath,amssymb,amsfonts}
\usepackage{algorithmic}
\usepackage{graphicx}
\usepackage{textcomp}
\usepackage{xcolor}
\def\BibTeX{{\rm B\kern-.05em{\sc i\kern-.025em b}\kern-.08em
    T\kern-.1667em\lower.7ex\hbox{E}\kern-.125emX}}
\begin{document}

\title{GPS in the Classroom\\
\thanks{Identify applicable funding agency here. If none, delete this.}
}

\author{\IEEEauthorblockN{Andrew Neish}
\IEEEauthorblockA{\textit{Aeronautics and Astronautics} \\
\textit{Stanford University}\\
Stanford, USA \\
amneish@stanford.edu}
\and
\IEEEauthorblockN{Tyler Reid}
\IEEEauthorblockA{
\textit{Xona Space Systems}\\
San Mateo, USA \\
tyler@xonaspace.com}
\and
\IEEEauthorblockN{Frank van Diggelen}
\IEEEauthorblockA{\textit{Android Location} \\
\textit{Google}\\
Mountain View, USA \\
fvandiggelen@google.com}
\and
\IEEEauthorblockN{Grace Gao}
\IEEEauthorblockA{\textit{Aeronautics and Astronautics} \\
\textit{Stanford University}\\
Stanford, USA \\
gracegao@stanford.edu}
}

\maketitle

\begin{abstract}
This is the abstract.
\end{abstract}

\begin{IEEEkeywords}
GPS
\end{IEEEkeywords}

\section{Introduction}


\begin{thebibliography}{00}
\bibitem{b1} G. Eason, B. Noble, and I. N. Sneddon, ``On certain integrals of Lipschitz-Hankel type involving products of Bessel functions,'' Phil. Trans. Roy. Soc. London, vol. A247, pp. 529--551, April 1955.

\end{thebibliography}
\vspace{12pt}


\end{document}
